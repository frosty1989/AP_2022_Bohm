%%Nevynechávat volný řádek

Laser shock peening is a surface treatment process used to improve the physico-chemical characteristics (fatigue life, corrosion resistance) of metallic components. Laser shock peening induces residual stresses beneath the treated surface of metallic materials. Laser shock peening applications have been rising in recent years, mainly due to the ever-increasing energy and decreasing prices of laser systems with parameters suitable for this treatment process. This study seeks to solve two problems: how can RoboDK with its Python application user interface be effectively used to generate robotic arm programs for laser shock peening applications, and what will be the future of the PhD thesis based on this study. The first problem was solved by modifying an existing RoboDK processor. For the second problem, the PhD thesis's future direction will be directed towards a LabVIEW-based control system. The created solution provides the possibility of generating robotic arm programs specially tailored for laser shock peening applications. One of the main contributions of this work is to simplify the generation of robotic arm programs for parts with complex geometries.


