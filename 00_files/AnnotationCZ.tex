%%Nevynechávat volný řádek
Laser shock peening je proces povrchové úpravy používaný ke zlepšení fyzikálně-chemických vlastnosti (únavová životnost, odolnost proti korozi) kovových součástí. Laser shock peening vyvolává pod upraveným povrchem kovových materiálů zbytková napětí. Aplikace procesu laser shock peening v posledních letech rostou, a to především díky stále rostoucí energii a klesajícím cenám laserových systémů s parametry vhodnými pro tento proces. Tato studie se snaží vyřešit dva problémy: jak lze software RoboDK a jeho Python application user interface efektivně využít k vytváření robotických programů pro aplikaci laser shock peening a jaké bude budoucí směřování doktorské práce založené na této studii. První problém byl vyřešen úpravou stávajícího procesoru RoboDK. V případě druhého problému bude budoucí směřování doktorské práce zaměřeno na řídicí systém založený na softwaru LabVIEW. Vytvořené řešení poskytuje možnost generovat programy pro robotické rameno speciálně uzpůsobené pro aplikaci laser shock peening. Jedním z hlavních přínosů práce je zjednodušení generování programů robotického ramene pro díly se složitou geometrií.


