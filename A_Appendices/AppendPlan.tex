\chapter{Časový plán absolventské práce \label{ch:ApendPLAN}}


\begin{center}
\begin{tabular}{|p{8cm}|c|r|r|}
  \hline
  \textbf{\hspace{3.3cm} Činnost} & \textbf{Časová} & \textbf{Termín\,\,\,} & \textbf{Splněno\,} \\
  {} & \textbf{náročnost} & \textbf{ukončení} & {} \\
  \hline

  %Vyplňte následující tabulku - text pište mezi složené závorky { }
  {objednávka elektrických komponent}
                & {2 týdny} & \,{15.04.2010} & {03.05.2010} \\\hline
  {dodání elektrických komponent}
                & {4 týdny} & \,{15.05.2010} & {} \\\hline
  {výkresová dokumentace}
                & {2 měsíce} & \,{10.10.2010} & {17.12.2010} \\\hline
  {stavba modelu -- lepení}
                & {2 měsíce} & \,{20.02.2011} & {} \\\hline
  {návrh plošných spojů}
                & {3 týdny} & {10.02.2011} & {} \\\hline
  {výroba a osazení plošných spojů}
                & {3 týdny} & {05.03.2011} & {} \\\hline
  {AP: kapitola Úvod}
                & {2 týdny} & {10.03.2011} & {} \\\hline
  {AP: kompletní text}
                & {} & {30.03.2011} & {} \\\hline
\end{tabular}
\end{center}





\textcolor{red}{\em Tato příloha je povinnou přílohou AP! Nebylo by od věci, znázornit plán pomocí úsečkového diagramu.\/}

Časový plán dobře promyslete (s~vedoucím AP) a~vyplňte \textbf{ještě v~rámci předmětu SAP ve druhém ročníku}. Časová náročnost je doba, jakou plánujete pracovat na dané činnosti. Termín ukončení je plánovaný termín, kdy hodláte danou činnost dokončit. Splněno je termín, kdy jste činnost poté skutečně ukončili.

\textcolor{blue}{\em Přílohy dále typicky obsahují technické údaje, dokumentace, prospekty apod., kte\-ré netvoří hlavní část práce a~ani je nemusel vytvořit autor práce.\/}

\textcolor{blue}{\em Tento text se nachází na konci souboru \texttt{A\_Appendices/AppendPlan.tex}. V~momentě, kdy výše uvedené provedete, tento barevný text vymažte.\/} 
